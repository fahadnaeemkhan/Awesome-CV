%-------------------------------------------------------------------------------
%	SECTION TITLE
%-------------------------------------------------------------------------------
\cvsection{Work Experience}


%-------------------------------------------------------------------------------
%	CONTENT
%-------------------------------------------------------------------------------
\begin{cventries}

%---------------------------------------------------------
  \cventry
    {Senior Staff Software Engineer} % Job title
    {Equinix} % Organization
    {Redwood city, CA} % Location
    {March. 2021 - Present} % Date(s)
    {
      \begin{cvitems} % Description(s) of tasks/responsibilities
        \item {Contributed to design of Equinix Network Observability platform (ETO) from picking the tech-stack to deployment on k8 using ArgoCD}
        \item {Fixed bugs in opensource softwares that were used in ETO e.g. gNMIC-server}
        \item {Integrated ETO with TRIX (which provides newtwork observability equinixmetal) by which ETO got graphQL API interface}
        \item {Extended TRIX to work with Time Series Data Bases (TSDB)}
        \item {Added support for Arrcus (new vendor) in TRIX}
        \item {Wrote TSDB interface library (in Go) with implementation for mimir, prometheus and influxDB. Interface allows applications to work with TSDBs}
        \item {Wrote replay-stats script (in Go) to sync Time Series data b/w prometheus and influxDB.}
        \item {Wrote a cache layer (in Elixir) for ETO to correlate business logic}
        \item {Lead the design of gnmi sensor test automation platform (in python)}
        \item {Worked on open source, programable dataplane Vector Packet Processing (VPP), extended it to support feature like protocol independent NAT, bug fixes etc.
               All the code is committed back to the project}
        \item {Designed and developed Elixir based VPP agent that implements binary message passing protocol exposed by VPP.
               Agent exposes gRPC based northbound interface that is used by SDN controller to program VPP. Agent also listens to VPP events like interface state changes and
               takes appropriate action defined e.g. notify SDN controller etc}
        \item {Helped dataplane team in various VPP related blockers. Also helped the team with the right tooling for the project.}
        \item {Worked with dataplane team on various POCs e.g. VPP as L2 bridge connecting two KVM VMs, service chaining in VPP using VPP graph node abstraction i.e. ACL -> NAT -> Routing,
               VPP and hugepages configuration for performance tuning}
        \item {Briefly worked on P4 programable SmartNIC by Pensando for the POC of IPSEC offload}
        \item {Implemented Vault based mTLS authentication for one of the REST services}
      \end{cvitems}
    }

%---------------------------------------------------------
  \cventry
    {Staff Software Engineer} % Job title
    {Equinix} % Organization
    {Sunnyvale, CA} % Location
    {December. 2018 - March. 2021} % Date(s)
    {
      \begin{cvitems} % Description(s) of tasks/responsibilities
        \item {Leading team of 4 to 6 developers to provide software solutions on different problems}
        \item {Designed and developed test automation framework that helps Equinix certify different
               products like Equnix Fabric (EF), Network Edge (NE)(python)}
        \item {Developed execution portal for the framework to help team run the automation easily. Portal is written in Elixir using Phoenix LiveView}
        \item {Setup CI/CD for test automation framework}
      \end{cvitems}
    }

%---------------------------------------------------------
  \cventry
    {Software Engineer} % Job title
    {Equinix} % Organization
    {Sunnyvale, CA} % Location
    {July. 2017 - December. 2018} % Date(s)
    {
      \begin{cvitems} % Description(s) of tasks/responsibilities
        \item {Implemented driver for Ciena WaveServer and OpenLineSystem (OLS) controller MCP in ONOS.
               All the code is committed back to ONOS opensource community.}
        \item {Extended many components in ONOS e.g. CLI, Optical Intent Compiler, OAuth2
               authentication for REST protocol etc.}
        \item {Developed alarm-handler ONOS app. The app takes well defined
               actions base on the alarm e.g. installation/removal of intents if port is down.
               (This app is not part of ONOS repo)}
      \end{cvitems}
    }

%---------------------------------------------------------
  \cventry
    {Software Engineer} % Job title
    {Cisco System.Inc} % Organization
    {San Jose, CA} % Location
    {March. 2015 - July. 2017} % Date(s)
    {
      \begin{cvitems} % Description(s) of tasks/responsibilities
        \item {Designed and Developed automation framework in python.
                Framework is written in python using open source libraries which makes it easier for the customer to deploy.
                It has been developed using OOP techniques which allows end user to build platform independent and protocol agnostic test automation.}
        \item {Wrote backend libraries for the infra.}
        \item {Developed service model for Open DayLight (ODL) using YANG as data model, NETCONF and RESTCONF as transport protocol.
        		RESTCONF is used as northbound and NETCONF as southbound protocol.}
        \item {Code reviews.}
      \end{cvitems}
    }

%---------------------------------------------------------
  \cventry
    {MTS - Intern} % Job title
    {Open Networking Laboratory} % Organization
    {Melo Park, CA} % Location
    {May. 2014 - March 2015} % Date(s)
    {
      \begin{cvitems} % Description(s) of tasks/responsibilities
        \item {Worked on two open source SDN products OpenVirteX (OVX) and Open Networking Operating System (ONOS).}
        \item {Developed new features for OVX like VirtualIPAddressing, link and route recovery etc.}
        \item {Designed and developed OVXTesting Framework under the guidance of Ali Al-Shabibi.}
        \item {Developed CLI and APIs for ONOS-SegmentRouting under the guidance of Saurav Das.}
        \item {End to end Integration of packet-optical use case of ONOS.}
        \item {Learning Erlang, bug fixes and adding new feature to Linc-oe.}
        \item {Added features in Mininet to support Optical components.}
      \end{cvitems}
    }

\end{cventries}
