%-------------------------------------------------------------------------------
%	SECTION TITLE
%-------------------------------------------------------------------------------
\cvsection{Research Experience}


%-------------------------------------------------------------------------------
%	CONTENT
%-------------------------------------------------------------------------------
\begin{cventries}

%---------------------------------------------------------
  \cventry
    {MS-Thesis} % Job title
    {Wichita State University} % Organization
    {Wichita, KS} % Location
    {Aug. 2013 - Dec. 2014} % Date(s)
    {
      \begin{cvitems} % Description(s) of tasks/responsibilities
        \item {Innovation in SDN and Throughput Optimum Scheduling Algorithm}
      \end{cvitems}
    }

  \cventry
    {Research Assistant} % Job title
    {National University and Engineering \& Technology (NUST) H-12, Islamabad} % Organization
    {Islamabad, Pakistan} % Location
    {Aug. 2012 - July. 2013} % Date(s)
    {
      \begin{cvitems} % Description(s) of tasks/responsibilities
        \item {Learned SDN, OpenFlow and related technologies during the research work and
                gained experience on OpenFlow specification v1.0.}
        \item {Wrote many mininet scripts using mininet python API to implement different network topologies}
        \item {Programmed Pox controller to implement desired functionality such as load balancer, firewall
                and etc.}
        \item {Learned Pyretic and PyResonance under Nick Feamster online course as part of my research requirements
                and Implemented firewall using Pyretic.}
        \item {Worked on Open vswitch 1.7.1, using Socket programming (in C/C++) wrote server/client program that runs
                on two different (mininet virtual) hosts connected to Openvswitch (for the communication purposes).}
      \end{cvitems}
    }

  \cventry
    {Final Year Project Research (Undergrad) - FYP} % Job title
    {National University and Engineering \& Technology (NUST) H-12, Islamabad} % Organization
    {Islamabad, Pakistan} % Location
    {Aug. 2011 – Aug. 2012} % Date(s)
    {
		Mobile Visualblock Programing (MVP) is a drag and drop tool for the development of iPhone application.
		User just drag and drop puzzle like object and connect them to implement their logic and at the backend
		our tool generates its Objective-C code for the iPhone.
		It helps new developer to concentrate more on logic then the complex syntax of Objective-C.
		We as member of two successfully implemented this as our Final year Project. The project was implemented using 
		Java, XML, apple scripting and Objective C. Initially Java library “open blocks was used to implement the interface,
		XML file is created by open blocks which is used to generate objective-C code. We also wrote XML to Objective C converter.
		We modified open blocks to create desired XML files an part of implementation. Using apple script we automated
		code compilation through Xcode.
    }

\end{cventries}
